%\mag1440
%\mag600
%\documentclass[draft,a4paper,12pt,reqno,oneside]{amsart}
%\documentclass[final,a4paper,12pt,reqno,oneside]{amsart, extarticle}
\documentclass[final,a4paper,14pt,reqno,oneside]{extarticle}
%\documentclass[draft,a4paper,12pt,reqno]{amsart}
%\documentclass[12pt]{article}
%\usepackage[T1]{fontenc}
\usepackage{cmap}
\usepackage[utf8]{inputenc}
\usepackage[T2A]{fontenc}
\usepackage[T2B]{fontenc}
\usepackage[T2C]{fontenc}
\usepackage[russian]{babel}
%\input glyphtounicode
%\pdfgentounicode=1
\usepackage{amsmath}
\usepackage{amssymb}
\usepackage{verbatim}
\usepackage{wasysym}
\usepackage{longtable}
\usepackage[center]{titlesec}
%\usepackage{sectsty}
%\usepackage{epic}
%\usepackage{eepic}
\usepackage{epsfig}
%\usepackage{floatflt}
\usepackage{graphicx}
%\usepackage{chapterbib}
\usepackage[nottoc]{tocbibind}
\usepackage[russian]{cleveref}

%\newcommand{\crefmiddleconjunction}{, }
%\newcommand{\creflastconjunction}{ и~}
%\newcommand{\crefrangeconjunction}{--}
%\newcommand{\crefpairconjunction}{, }
\crefname{equation}{\!\!}{\!\!}
\crefname{figure}{\!\!}{\!\!}

\linespread{1.3}

\hoffset=-10mm
\textwidth=175 mm
\textheight=263 mm
\topmargin=-20 mm
\headheight=3 mm
\headsep=10 pt
\oddsidemargin=12 mm

%\setlength{\oddsidemargin}{5 mm} \setlength{\topmargin}{0 mm}
%\setlength{\headheight}{0 mm} \setlength{\headsep}{0 mm}
%\setlength{\textwidth}{160 mm} \setlength{\textheight}{240 mm}

%\tolerance=1000
%\pagestyle{empty}

\graphicspath{{./images/}}
 


%\DeclareMathAccent{\widetilde}{\mathord}{largesymbols}{"65}
%\DeclareMathAccent{\widetilde}{\mathrel}{largesymbols}{93}
%\DeclareMathAccent{\widetilde}{\mathrel}{largesymbols}{"12}
%\DeclareMathAccent{\widetilde}{\mathord}{letters}{"5F}
%\DeclareMathAccent{\widetilde}{\mathalpha}{AMSa}{"61}
\DeclareMathAccent{\widetilde}{\mathalpha}{largesymbols}{"45}
%\DeclareMathAccent{\widehat}{\mathord}{largesymbols}{"62}
\newcommand\ff{\varphi}
\renewcommand{\f}{\varphi}
\newcommand\eps{\varepsilon}
%\newcommand{\e}{\varepsilon}
\newcommand{\Q}{\theta}
\newcommand{\la}{\lambda}
\newcommand{\al}{\alpha}
\newcommand{\be}{\beta}
\newcommand{\ga}{\gamma}
\newcommand{\s}{\sigma}
\newcommand{\x}{\xi}
\newcommand{\z}{\zeta}
\renewcommand{\r}{\rho}
\newcommand{\n}{\eta}
\renewcommand{\t}{\tau}
\newcommand{\er}{\bar{e}_r}
\newcommand{\F}{\Phi}
\newcommand{\hU}{\hat{U}}
\renewcommand{\P}{\Psi}
\newcommand{\nab}{\nabla}
\newcommand{\Lap}{\Delta}
\newcommand\om{\omega}
\newcommand\Om{\Omega}
\newcommand\Gmm{\Gamma}
\renewcommand{\k}{\varkappa}
\newcommand\dss{\displaystyle}
\newcommand\fr{\frac}
\newcommand\df{\dfrac}
\newcommand\de{\partial}
\newcommand\Op{\operatorname}
\newcommand\idot{\,\cdot}
\renewcommand{\.}{\,\cdot\,} % index dot
\newcommand{\dotm}{\!\cdot\!} % dot for scalar multiple
\newlength{\ertp}
%\renewcommand{\No}{№}
%\renewcommand{\sectname}{Лекция }\sectname
%\renewcommand{\cite}[1]{}
%\renewcommand{\"}{\symbol{34}}
\def\q{\quad}
\def\qq{\qquad}
%\DeclareMathOperator\div{div}
%\DeclareMathOperator\det{det}
\DeclareMathOperator{\e}{e}
\DeclareMathOperator{\diver}{div}
\DeclareMathOperator{\grad}{grad}
\DeclareMathOperator{\rot}{rot}
\DeclareMathOperator{\dif}{d}
\newcommand\diff{\dif \!}
\DeclareMathOperator{\opL}{L}
\DeclareMathOperator{\T}{T}
\DeclareMathOperator{\const}{const}
\unitlength 1.0mm \linethickness{0.4pt}


\newcommand\equationF[3]{%
\begin{equation}{\label{#1}}
 \raisebox{#3pt}{\includegraphics[scale=#2]{FIGs/#1}}\ ,
\end{equation}
}%

\newcommand{\equationFnl}[3]{%
\begin{equation*}{\label{#1}}
 \raisebox{#3pt}{\includegraphics[scale=#2]{FIGs/#1}}\ ,
\end{equation*}
}%

\newcommand\equationFF[3]{$$\raisebox{#1pt}{#3}\eqno{(#2)}$$}%

\renewcommand\thesection{\arabic{section}.}
\renewcommand\thesubsection{\thesection\arabic{subsection}.}
\renewcommand\thesubsubsection{\thesubsection\arabic{subsubsection}.}
%\allsectionsfont{\centering}
\begin{document}

\makeatletter
%\renewcommand{\thesection}{}
\renewcommand{\@oddhead}{}
\renewcommand{\@oddfoot}{\hfil \thepage \hfil}
\renewcommand{\l@section}{\@dottedtocline{1}{0em}{2.3em}} %содержание
%\renewcommand{\l@section}[2]{\hbox to\textwidth{#1\dotfill #2}}
\makeatother

%\renewcommand{\bibname}{Список литературы к лекции}
%\renewcommand\refname{Какой-то список}

\newcommand{\ssection}[1]{%
  \section[#1]{\centering\normalfont\scshape #1}}
\newcommand{\ssubsection}[1]{%
  \subsection[#1]{\raggedright\normalfont\itshape #1}}

\setlength{\parindent}{1.0cm}

\setlength{\leftmargini}{1.0cm}
\def\theenumi{\arabic{enumi}}
\def\labelenumi{\theenumi)}

%\setlength{\par1}{\parindent}
%\setlength{\parskip}{1ex}
%\parskip=2pt\parindent 0pt
\setlength{\ertp}{\parindent}

%Замечания Толоконникова
\newcommand{\todo}{\phantom} 
%зАКОММЕНТИРОВАТЬ!!!!
%\newcommand{\todo}{\par \textbf}

%Мои обозначения:
%Внешний радиус цилиндра R_2
%Плотность материала цилиндра r_2
%Упругие постоянные цилиндра Lambda_2, mu_2
%Радиус полости R_1
%Внешний радиус слоя R_3
%Система координат цилиндра r_2, phi_2, z_2
%Система координат полости r_1, phi_1, z_1
%Модули упругости слоя Lambda_3, mu_3
%Плотность слоя pho
%Плотность окружающей цилиндр жидкости pho_1
%Скорость звука в окружающей цилиндр жидкости c_1
%Плотность находящейся в полости жидкости pho_2
%Плотность находящейся в полости жидкости c_2

\newcommand{\en}{_e} % среда
\renewcommand{\o}{_o} % падающая
\renewcommand{\sc}{_s} % рассеянная
\renewcommand{\c}{_\circ} % полость
\newcommand{\cyl}{_\odot} % цилиндр
\renewcommand{\l}{_\circledcirc} % слой
\newcommand{\prodol}{_\leftrightarrow} % продольная волна
\newcommand{\popr}{_\updownarrow} % поперечная волна



\renewcommand{\bibname}{СПИСОК ИСПОЛЬЗОВАННЫХ ИСТОЧНИКОВ}
\renewcommand\refname{СПИСОК ИСПОЛЬЗОВАННЫХ ИСТОЧНИКОВ}

%\input{title.tex}

%\newpage
\setcounter{page}{2}
\thispagestyle {empty}
\renewcommand{\contentsname}{\centering СОДЕРЖАНИЕ}
\tableofcontents

\newpage
\section*{ВВЕДЕНИЕ}
\addcontentsline{toc}{section}{ВВЕДЕНИЕ}
\todo{Акустика, применени акустики}

Существуют разлиные подходы к изменению звукоотражающих характеристик тел в определенных направлениях. Изменение характеристик рассеяния звука упругих тел можно осуществить с помощью специальных покрытий. Представляет интерес исследовать звукоотражающие свойства тел с покрытиями в виде непрерывно неоднородного упругого слоя. Такой слой легко реализовать с помощью системы тонких однородных упругих слоев с различными значениями механических параметров (плотности и упругих постоянных).

В настоящей работе решается задача о рассеянии плоской монохраматической звуковой волны, падающей наклонно на упругий круговой цилиндр с неконцентрической полостью, покрытый радиально-неоднородным упругим слоем.

\newpage
\section{МАТЕМАТИЧЕСКОЕ МОДЕЛИРОВАНИЕ}
\todo{Распространие звуковых волн}

\newpage
\subsection{Распространение звука в идеальной жидкости}
\todo{Из книги Толоконников, Ларин или из лабы}

\newpage
\subsection{Распространение звуковых волн в упругих телах}
\todo{Переделать для упругих тел. Взять из Ландау-Лифшиц Теория упругости, Амензаде Теория упругости. Все в своих системах координат. Решение в специальных функциях.}

\newpage
\section{ДИФРАКЦИЯ ЗВУКОВЫХ ВОЛН НА УПРУГОМ ЦИЛИНДРЕ, ИМЕЮЩЕМ ПРОИЗВОЛЬНО РАСПОЛОЖЕННУЮ ПОЛОСТЬ И НЕОДНОРОДНОЕ ПОКРЫТИЕ}

\newpage
\subsection{Обзор литературы по проблеме исследования}
\todo{Взять у Филатовой + новые работы. Известия ТулГУ №3 - новые статьи. Статьи Ларина, Скобельцына, Толоконникова.}

\newpage
\subsection{Постановка задачи} 
\todo{Рисунок и рассмотрим, пусть. И закончить Требуется найти волновые поля в упругом теле.}

Рассмотрим бесконечный однородный упругий цилиндр с внешним радиусом $R\cyl,$ материал которого характеризуется плотностью $p\cyl$ и упругими постоянными $\lambda\cyl$ и $\mu\cyl.$ Цилиндр имеет произвольно расположенную цилиндрическую полость с радиусом $R\c.$ Оси цилиндра и полости являются параллельными. Цилиндр имеет покрытие в виде неоднородного изотропного упругого слоя, внешний радиус которого равен $R\l.$ Для решения задачи ввдем цилиндрические системы координат $\r\cyl, \f\cyl, z\cyl$ и $\r\c, \f\c, z\c,$ связанные с цилиндром и его полостью соответственно.

Полагаем, что модули упругости $\lambda\l$ и $\mu\l$ материала неоднородного цилиндрического слоя опиcываются дифференциируемыми функциями цилиндрической радиальной координаты $\r\cyl,$ а плотность $p\l$ -- непрерывной функцией координаты~$\r\cyl.$ 

Будем считать, что окружающая цилиндр и находящаяся в его полости жидкость являются идеальными и однородными, имеющими в невозмущенном состоянии плотности $\r\en, \r\c$ и скорости звука $c\en, c\c$ соответственно.

Пусть из внешнего пространства на цилиндр произвольным образом падает плоская звуковая волна, потенциал скоростей которой равен
$$\P\o=A\o \exp\{i[(\bar{k}\en)\cdot \bar{r}\o)-\omega\en t]\},$$
где $A\o$ -- амплитуда волны; $\bar{k}\en$ -- волновой вектор падающей волны; $\bar{r}\o$ -- радиус-вектор; $\omega\en$ -- круговая частота. В дальнейшем временной множитель $\exp\{-i\omega\en t\}$ будем опускать.

В цидиндрической системе координат падающая волна запишется в виде
$$\P\o=A\o\exp\{ik\en[r\o\sin{\hat{\theta}\en}\cos(\f\en-\hat{\f}\en)+z\cos\hat{\Q}\en]\},$$
где $\hat{\theta}\en$ и $\hat{\f}\en$ -- полярный и азимутальный углы падения волны; $k\en=\omega\en / c\en$ -- волновое число во внешней области.

Определим отраженную от цилиндра волну и возбужденную в его полости звуковые волны, а также найдем поля смещений в упругом цилиндре и неоднородном слое.

\newpage
\subsection{Аналитическое решение задачи}

Потенциал скоростей падающей плоской волны представим в виде
$$\P_0(r_1, \f_1, z_1)=A_0\exp\{i\alpha_1 z_1\}\sum_{n=-\infty}^{\infty}i^nJ_n(\beta_1 r_1)\exp\{in(\f_1-\f_{1_0})\},$$
где $J_n(x)$ -- цилиндрическая функция Бесселя порядка $n;$ $\alpha_1=k_1\cos\theta_0;$ \\$\beta_1=k_1\sin\theta_0.$

В установившемся режиме колебаний задача определения акустических полей вне цилиндра и внутри его полости заключается в нахожждении решений уравнения Гельмгольца
\begin{align}
&\Lap\P_1+k_1^2\P_1=0,\\
&\Lap\P_2+k_2^2\P_2=0\label{eq_gel_for_enviroment},
\end{align}
где $\P_1$ -- потенциал скоростей акустического поля в полости цилиндра;\\ $k_1=\frac{\omega}{c_1}$ -- волновое число жидкочти в полости цилиндра;
$\P_{2}$ -- потенциал скоростей полного акустического поля во внешней среде. 

В силу линейной постановки задачи
\begin{equation}
\P_2=\P_i+\P_s,
\end{equation}
где $\P_s$ -- потенциал скоростей рассеянной звуковой волны.

Тогда из \eqref{eq_gel_for_enviroment} получаем уравнение для нахождения $\P_s:$
\begin{equation}\label{eq__gel_for_psi_s}
\Lap\P_s+k_s^2\P_s=0.
\end{equation}

Уравнения \eqref{eq__gel_for_psi_s} и \eqref{eq_gel_for_enviroment} запишем в цилиндрических системах координат $r_1, \f_1, z_1$ и $r_2, \f_2, z_2$ соответственно. 

Отраженная волна $\P_s$ должна удовлетворять условиям излучения на бесконечности, а звуковая волна в полости цилиндра $\P_1$ -- условию ограниченности.

Поэтому потенциалы $\P_s$ и $\P_1$ будем искать в виде
\begin{align}
&\P_s(r_2, \f_2, z_2)= \exp\{i\alpha_2 z_2\}\sum_{n=-\infty}^{\infty}A_nH_n(\beta_2 r_2)\exp\{in(\f_2-\f_{2_0})\},\\
&\P_1(r_1, \f_1, z_1)= \exp\{i\alpha_1 z_1\}\sum_{n=-\infty}^{\infty}B_nH_n(\beta_1 r_1)\exp\{in(\f_1-\f_{1_0})\},
\end{align}
где $H_n(x)$ -- цилиндрическая функция Ханкеля первого рода порядка $n.$

Скорости частиц жидкости и акустические давления вне $(j=2)$ и внутри $(j=1)$ цилиндра определяются по следующим формулам соответственно:
\begin{equation}
\bar{\nu}_j=\grad\P_i;\:\:p_j=i\r_j\omega\P_j\:\:\:(j=1,2).
\end{equation}

Распространение малых возмущений в упругом теле для установившегося режима движения частиц тела описывается скалярным и векторным уравнениями Гельмгольца:
\begin{align}
&\Lap\P+k_l^2\P=0,\\
&\Lap\bar{\F}+k_{\tau}^2\bar{\F}=0,\label{fi_gel}
\end{align}
где $k_l=\omega/c_l$ и $k_{\tau}=\omega/c_{\tau}$ -- волновые числа продольных и поперечных упругих волн соответственно; $\P$ и $\bar\F$ -- скалярный и векторный потенциалы смещения соответственно; $c_l=\sqrt{(\lambda_1+2\mu_1)/\r_2}$ и $c_{\tau}=\sqrt{\mu_1/\r_2}$ -- скорости продольных и поперечных волн соответственно.

При этом вектор смещения $\bar{u}$ представляется в виде:
\begin{equation}
\bar{u}=\grad\P+\rot\bar{\F}.
\end{equation}

Векторное уравнение \eqref{fi_gel} в цилиндрической системе координат в обшем случае не распадается на три независимых скалярных уравнения относительно проекций вектора $\bar{\F},$ а представляет собой систему трех уравнений, решение которой сопряжено со значительными математическими трудностями.

Представим вектор $\F$ в виде
$$\F=\rot(L\bar{e}_z)+\frac1{k_\tau}\rot\rot(M\bar{e}_z)=\rot(L\bar{e}_z)+k_\tau M\bar{e}_z+\frac1{k_\tau}\grad\biggl(\frac{\partial M}{\partial z}\biggr),$$
где $L$ и $M$ -- скалярные функции пространственных координат $r, \phi, z;$ $\bar{e}_z$ -- единичный вектор оси z.

Тогда векторное уравнение \eqref{fi_gel} заменится двумя скалярными уравнениями Гельмгольца относительно функций $L$ и $M$
\begin{align*}
&\Lap L+k_{\tau}^2L=0,\\
&\Lap M+k_{\tau}^2M=0.
\end{align*}

С учетом условия ограниченности функции $\P, L$ и $M$ будем искать в виде
\begin{align}
&\P(r, \f, z)= \exp\{i\alpha z\}\sum_{n=-\infty}^{\infty}C_nH_n(k_1 r)\exp\{in(\f-\f_0)\},\\
&L(r, \f, z)= \exp\{i\alpha z\}\sum_{n=-\infty}^{\infty}D_nH_n(k_2 r)\exp\{in(\f-\f_0)\},\\
&M(r, \f, z)= \exp\{i\alpha z\}\sum_{n=-\infty}^{\infty}E_nH_n(k_2 r)\exp\{in(\f-\f_0)\},
\end{align}
где $k_1=\sqrt{k_l^2-\alpha^2}, k_2=\sqrt{k_{\tau}^2-\alpha^2}.$

Компоненты вектора смещения $\bar{u},$ записанные через функции $\P, L$ и $M$ в цилиндрической системе координат, имеют вид
\begin{align}
&u_r=\frac{\partial \P}{\partial r}+\frac{\partial^2 L}{\partial r\partial z}+\frac{k_\tau} {r}\frac{\partial M}{\partial \f},\\
&u_\f=\frac{1}{r}\frac{\partial \P}{\partial \f}+\frac{1}{r}\frac{\partial^{2} L}{\partial\f\: \partial z}-k_{\tau}\frac{\partial M}{\partial r},\\
&u_z=\frac{\partial\P}{\partial z}-\frac{\partial^{2} L}{\partial r^{2}}-\frac{1}{r}\frac{\partial L}{\partial r}-\frac{1}{r^2}\frac{\partial^{2}L}{\partial\f^{2}}.
\end{align}

Соотношения между компонентами тензора напряжений $\sigma_{ij}$ и вектора смещения $\bar{u}$ в однородном изотропном упругом цилиндре записываются следующим образом:
\begin{align}
&\sigma_{rr}=\lambda\Biggl(\frac{\partial u_r}{\partial r}+\frac{1}{r}\biggl(\frac{\partial u_\f}{\partial\f}+u_r\biggr)+\frac{\partial u_z}{\partial z}\Biggr)+2\:\mu\:\frac{\partial u_r}{\partial r},\\
&\sigma_{\f\f}=\lambda\Biggl(\frac{\partial u_r}{\partial r}+\frac{1}{r}\biggl(\frac{\partial u_\f}{\partial\f}+u_r\biggr)+\frac{\partial u_z}{\partial z}\Biggr)+2\:\mu\:\biggl(\frac{1}{r}\frac{\partial u_\f}{\partial\f}+\frac{u_r}{r}\biggr),\\
&\sigma_{zz}=\lambda\Biggl(\frac{\partial u_r}{\partial r}+\frac{1}{r}\biggl(\frac{\partial u_\f}{\partial\f}+u_r\biggr)+\frac{\partial u_z}{\partial z}\Biggr)+2\:\mu\:\frac{\partial u_z}{\partial z},\\
&\sigma_{r\f}=\mu\Biggl(\frac{1}{r}\frac{\partial u_r}{\partial\f}+\frac{\partial u_\f}{\partial r}-\frac{u_{\f}}{r}\Biggr),\\
&\sigma_{rz}=\mu\Biggl(\frac{\partial u_z}{\partial r}+\frac{\partial u_r}{\partial z}\Biggr),\\
&\sigma_{\f z}=\mu\Biggl(\frac{\partial u_\f}{\partial z}+\frac{1}{r}\frac{\partial u_z}{\partial \f}\Biggr).
\end{align}

Уравнения движения неоднородного изотропного упругого цилиндрического слоя в случае установившихся колебаний в цилиндрической системе координат имеют вид:
\begin{align}
\frac{\partial\sigma_{rr}}{
\partial r}+\frac{1}{r}\frac{\partial\sigma_{r\f}}{\partial\f}+\frac{\partial\sigma_{rz}}{\partial z}+\frac{\sigma_{rr}-\sigma_{\f\f}}{r}&=-\omega^2\rho(r)u_r,\\
\frac{\partial\sigma_{r\f}}{
\partial r}+\frac{1}{r}\frac{\partial\sigma_{\f\f}}{\partial\f}+\frac{\partial\sigma_{\f z}}{\partial z}+\frac{2}{r}\sigma_{r\f}&=-\omega^2\rho(r)u_{\f},\\
\frac{\partial\sigma_{rz}}{
\partial r}+\frac{1}{r}\frac{\partial\sigma_{\f z}}{\partial\f}+\frac{\partial\sigma_{zz}}{\partial z}+\frac{1}{r}\sigma_{rz}&=-\omega^2\rho(r)u_z.
\end{align}

\newpage
\subsection{Решение краевой задачи для системы обыкновенных дифференциальных уравнений}

\newpage
\section{ЧИСЛЕННЫЕ ИССЛЕДОВАНИЯ}

\newpage
\subsection{Диаграмма направленности}

\newpage
\subsection{Частотные характеристики}


\newpage
\section*{ЗАКЛЮЧЕНИЕ}
\addcontentsline{toc}{section}{ЗАКЛЮЧЕНИЕ}
Была поставлена задача о~дифракции плоских звуковых волн на~упругой сфере, имеющей произвольно расположенную полость и неоднородное покрытие. В~данной работе приведены основные уравнения колебаний, а также разложения в~ряд искомых функций для~внешней среды сферы, а также полости тела.

\newpage
\section*{ЛИТЕРАТУРА}
\addcontentsline{toc}{section}{ЛИТЕРАТУРА}

\newpage
\section*{ПРИЛОЖЕНИЕ}
\addcontentsline{toc}{section}{ПРИЛОЖЕНИЕ}
\end{document}