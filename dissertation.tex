%\mag1440
%\mag600
%\documentclass[draft,a4paper,12pt,reqno,oneside]{amsart}
%\documentclass[final,a4paper,12pt,reqno,oneside]{amsart, extarticle}
\documentclass[final,a4paper,14pt,reqno,oneside]{extarticle}
%\documentclass[draft,a4paper,12pt,reqno]{amsart}
%\documentclass[12pt]{article}
%\usepackage[T1]{fontenc}
\usepackage{cmap}
\usepackage[utf8]{inputenc}
\usepackage[T2A]{fontenc}
\usepackage[T2B]{fontenc}
\usepackage[T2C]{fontenc}
\usepackage[russian]{babel}
%\input glyphtounicode
%\pdfgentounicode=1
\usepackage{amsmath}
\usepackage{amssymb}
\usepackage{verbatim}
\usepackage{wasysym}
\usepackage{longtable}
\usepackage[center]{titlesec}
%\usepackage{sectsty}
%\usepackage{epic}
%\usepackage{eepic}
\usepackage{epsfig}
%\usepackage{floatflt}
\usepackage{graphicx}
%\usepackage{chapterbib}
\usepackage[nottoc]{tocbibind}
\usepackage[russian]{cleveref}

%\newcommand{\crefmiddleconjunction}{, }
%\newcommand{\creflastconjunction}{ и~}
%\newcommand{\crefrangeconjunction}{--}
%\newcommand{\crefpairconjunction}{, }
\crefname{equation}{\!\!}{\!\!}
\crefname{figure}{\!\!}{\!\!}

\linespread{1.3}

\hoffset=-10mm
\textwidth=175 mm
\textheight=263 mm
\topmargin=-20 mm
\headheight=3 mm
\headsep=10 pt
\oddsidemargin=12 mm

%\setlength{\oddsidemargin}{5 mm} \setlength{\topmargin}{0 mm}
%\setlength{\headheight}{0 mm} \setlength{\headsep}{0 mm}
%\setlength{\textwidth}{160 mm} \setlength{\textheight}{240 mm}

%\tolerance=1000
%\pagestyle{empty}

\graphicspath{{./images/}}
 


%\DeclareMathAccent{\widetilde}{\mathord}{largesymbols}{"65}
%\DeclareMathAccent{\widetilde}{\mathrel}{largesymbols}{93}
%\DeclareMathAccent{\widetilde}{\mathrel}{largesymbols}{"12}
%\DeclareMathAccent{\widetilde}{\mathord}{letters}{"5F}
%\DeclareMathAccent{\widetilde}{\mathalpha}{AMSa}{"61}
\DeclareMathAccent{\widetilde}{\mathalpha}{largesymbols}{"45}
%\DeclareMathAccent{\widehat}{\mathord}{largesymbols}{"62}
\newcommand\ff{\varphi}
\renewcommand{\f}{\varphi}
\newcommand\eps{\varepsilon}
%\newcommand{\e}{\varepsilon}
\newcommand{\Q}{\theta}
\newcommand{\la}{\lambda}
\newcommand{\al}{\alpha}
\newcommand{\be}{\beta}
\newcommand{\ga}{\gamma}
\newcommand{\s}{\sigma}
\newcommand{\x}{\xi}
\newcommand{\z}{\zeta}
\renewcommand{\r}{\rho}
\newcommand{\n}{\eta}
\renewcommand{\t}{\tau}
\newcommand{\er}{\bar{e}_r}
\newcommand{\F}{\Phi}
\newcommand{\hU}{\hat{U}}
\renewcommand{\P}{\Psi}
\newcommand{\nab}{\nabla}
\newcommand{\Lap}{\Delta}
\newcommand\om{\omega}
\newcommand\Om{\Omega}
\newcommand\Gmm{\Gamma}
\renewcommand{\k}{\varkappa}
\newcommand\dss{\displaystyle}
\newcommand\fr{\frac}
\newcommand\df{\dfrac}
\newcommand\de{\partial}
\newcommand\Op{\operatorname}
\newcommand\idot{\,\cdot}
\renewcommand{\.}{\,\cdot\,} % index dot
\newcommand{\dotm}{\!\cdot\!} % dot for scalar multiple
\newlength{\ertp}
%\renewcommand{\No}{№}
%\renewcommand{\sectname}{Лекция }\sectname
%\renewcommand{\cite}[1]{}
%\renewcommand{\"}{\symbol{34}}
\def\q{\quad}
\def\qq{\qquad}
%\DeclareMathOperator\div{div}
%\DeclareMathOperator\det{det}
\DeclareMathOperator{\e}{e}
\DeclareMathOperator{\diver}{div}
\DeclareMathOperator{\grad}{grad}
\DeclareMathOperator{\rot}{rot}
\DeclareMathOperator{\dif}{d}
\newcommand\diff{\dif \!}
\DeclareMathOperator{\opL}{L}
\DeclareMathOperator{\T}{T}
\DeclareMathOperator{\const}{const}
\unitlength 1.0mm \linethickness{0.4pt}


\newcommand\equationF[3]{%
\begin{equation}{\label{#1}}
 \raisebox{#3pt}{\includegraphics[scale=#2]{FIGs/#1}}\ ,
\end{equation}
}%

\newcommand{\equationFnl}[3]{%
\begin{equation*}{\label{#1}}
 \raisebox{#3pt}{\includegraphics[scale=#2]{FIGs/#1}}\ ,
\end{equation*}
}%

\newcommand\equationFF[3]{$$\raisebox{#1pt}{#3}\eqno{(#2)}$$}%

\renewcommand\thesection{\arabic{section}.}
\renewcommand\thesubsection{\thesection\arabic{subsection}.}
\renewcommand\thesubsubsection{\thesubsection\arabic{subsubsection}.}
%\allsectionsfont{\centering}
\begin{document}

\makeatletter
%\renewcommand{\thesection}{}
\renewcommand{\@oddhead}{}
\renewcommand{\@oddfoot}{\hfil \thepage \hfil}
\renewcommand{\l@section}{\@dottedtocline{1}{0em}{2.3em}} %содержание
%\renewcommand{\l@section}[2]{\hbox to\textwidth{#1\dotfill #2}}
\makeatother

%\renewcommand{\bibname}{Список литературы к лекции}
%\renewcommand\refname{Какой-то список}

\newcommand{\ssection}[1]{%
  \section[#1]{\centering\normalfont\scshape #1}}
\newcommand{\ssubsection}[1]{%
  \subsection[#1]{\raggedright\normalfont\itshape #1}}

\setlength{\parindent}{1.0cm}

\setlength{\leftmargini}{1.0cm}
\def\theenumi{\arabic{enumi}}
\def\labelenumi{\theenumi)}

%\setlength{\par1}{\parindent}
%\setlength{\parskip}{1ex}
%\parskip=2pt\parindent 0pt
\setlength{\ertp}{\parindent}

%Замечания Толоконникова
\newcommand{\todo}{\phantom} 
%зАКОММЕНТИРОВАТЬ!!!!
%\newcommand{\todo}{\par \textbf}

%Мои обозначения:
%Внешний радиус цилиндра R_2
%Плотность материала цилиндра r_2
%Упругие постоянные цилиндра Lambda_2, mu_2
%Радиус полости R_1
%Внешний радиус слоя R_3
%Система координат цилиндра r_2, phi_2, z_2
%Система координат полости r_1, phi_1, z_1
%Модули упругости слоя Lambda_3, mu_3
%Плотность слоя pho
%Плотность окружающей цилиндр жидкости pho_1
%Скорость звука в окружающей цилиндр жидкости c_1
%Плотность находящейся в полости жидкости pho_2
%Плотность находящейся в полости жидкости c_2

\newcommand{\en}{_e} % среда
\renewcommand{\o}{_o} % падающая
\renewcommand{\sc}{_s} % рассеянная
\renewcommand{\c}{_\circ} % полость
\newcommand{\cyl}{_\odot} % цилиндр
\renewcommand{\l}{_\circledcirc} % слой
\newcommand{\prodol}{_\leftrightarrow} % продольная волна
\newcommand{\popr}{_\updownarrow} % поперечная волна


 
\renewcommand{\bibname}{СПИСОК ИСПОЛЬЗОВАННЫХ ИСТОЧНИКОВ}
\renewcommand\refname{СПИСОК ИСПОЛЬЗОВАННЫХ ИСТОЧНИКОВ}

%\input{title.tex}

%\newpage
\setcounter{page}{2}
\thispagestyle {empty}
\renewcommand{\contentsname}{\centering СОДЕРЖАНИЕ}
\tableofcontents

\newpage
\section*{ВВЕДЕНИЕ}
\addcontentsline{toc}{section}{ВВЕДЕНИЕ}
\todo{Акустика, применени акустики}

Существуют разлиные подходы к изменению звукоотражающих характеристик тел в определенных направлениях. Изменение характеристик рассеяния звука упругих тел можно осуществить с помощью специальных покрытий. Представляет интерес исследовать звукоотражающие свойства тел с покрытиями в виде непрерывно неоднородного упругого слоя. Такой слой легко реализовать с помощью системы тонких однородных упругих слоев с различными значениями механических параметров (плотности и упругих постоянных).

В настоящей работе решается задача о рассеянии плоской монохраматической звуковой волны, падающей наклонно на упругий круговой цилиндр с неконцентрической полостью, покрытый радиально-неоднородным упругим слоем.

\newpage
\section{МАТЕМАТИЧЕСКОЕ МОДЕЛИРОВАНИЕ}
\todo{Распространие звуковых волн}

\newpage
\subsection{Распространение звука в идеальной жидкости}
\todo{Из книги Толоконников, Ларин или из лабы}

\newpage
\subsection{Распространение звуковых волн в упругих телах}
\todo{Переделать для упругих тел. Взять из Ландау-Лифшиц Теория упругости, Амензаде Теория упругости. Все в своих системах координат. Решение в специальных функциях.}

\newpage
\section{ДИФРАКЦИЯ ЗВУКОВЫХ ВОЛН НА УПРУГОМ ЦИЛИНДРЕ, ИМЕЮЩЕМ ПРОИЗВОЛЬНО РАСПОЛОЖЕННУЮ ПОЛОСТЬ И НЕОДНОРОДНОЕ ПОКРЫТИЕ}

\newpage
\subsection{Обзор литературы по проблеме исследования}
\todo{Взять у Филатовой + новые работы. Известия ТулГУ №3 - новые статьи. Статьи Ларина, Скобельцына, Толоконникова.}

\newpage
\subsection{Постановка задачи} 
\todo{Рисунок и рассмотрим, пусть. И закончить Требуется найти волновые поля в упругом теле.}

Рассмотрим бесконечный однородный упругий цилиндр с внешним радиусом $R\cyl,$ материал которого характеризуется плотностью $p\cyl$ и упругими постоянными $\lambda\cyl$ и $\mu\cyl.$ Цилиндр имеет произвольно расположенную цилиндрическую полость с радиусом $R\c.$ Оси цилиндра и полости являются параллельными. Цилиндр имеет покрытие в виде неоднородного изотропного упругого слоя, внешний радиус которого равен $R\l.$ Для решения задачи ввдем цилиндрические системы координат $\r\cyl, \f\cyl, z\cyl$ и $\r\c, \f\c, z\c,$ связанные с цилиндром и его полостью соответственно.

Полагаем, что модули упругости $\lambda\l$ и $\mu\l$ материала неоднородного цилиндрического слоя опиcываются дифференциируемыми функциями цилиндрической радиальной координаты $\r\cyl,$ а плотность $p\l$ -- непрерывной функцией координаты~$\r\cyl.$ 

Будем считать, что окружающая цилиндр и находящаяся в его полости жидкость являются идеальными и однородными, имеющими в невозмущенном состоянии плотности $\r\en, \r\c$ и скорости звука $c\en, c\c$ соответственно.

Пусть из внешнего пространства на цилиндр произвольным образом падает плоская звуковая волна, потенциал скоростей которой равен
$$\P\o=A\o \exp\{i[(\bar{k}\en)\cdot \bar{r}\cyl)-\omega t]\},$$
где $A\o$ -- амплитуда волны; $\bar{k}\en$ -- волновой вектор падающей волны; $\bar{r}\cyl$ -- радиус-вектор; $\omega$ -- круговая частота. В дальнейшем временной множитель $\exp\{-i\omega t\}$ будем опускать.

В цилиндрической системе координат падающая волна запишется в виде
$$\P\o=A\o\exp\{ik\en[\r\cyl\sin{\hat{\theta}\en}\cos(\f\cyl-\hat{\f}\en)+z\cyl\cos\hat{\Q}\en]\},$$
где $\hat{\theta}\en$ и $\hat{\f}\en$ -- полярный и азимутальный углы падения волны; $k\en=\omega / c\en$ -- волновое число во внешней области.

Определим отраженную от цилиндра волну и возбужденную в его полости звуковые волны, а также найдем поля смещений в упругом цилиндре и неоднородном слое.

\newpage
\subsection{Аналитическое решение задачи}

Потенциал скоростей падающей плоской волны представим в виде
$$\P\o(\r\cyl, \f\cyl, z\cyl)=A_0\exp\{i\alpha  z\cyl\}\sum_{n=-\infty}^{\infty}i^nJ_n(\beta\en  \r\cyl)\exp\{in(\f\cyl-\hat\f\en)\},$$
где $J_n(x)$ -- цилиндрическая функция Бесселя порядка $n;$ $\alpha=k\en\cos\hat\theta\en;$ \\$\beta\en=\sqrt{k_{\en}^2-\alpha^2}.$

В установившемся режиме колебаний задача определения акустических полей вне цилиндра и внутри его полости заключается в нахождении решений уравнений Гельмгольца
\begin{align}
&\Lap\P\c+k\c^2\P\c=0\label{eq_gel_for_polost},\\
&\Lap\P\en+k\en^2\P\en=0\label{eq_gel_for_enviroment},
\end{align}
где $\P\c$ -- потенциал скоростей акустического поля в полости цилиндра;\\ $k\c=\frac{\omega}{c\c}$ -- волновое число жидкости в полости цилиндра;
$\P\en$ -- потенциал скоростей полного акустического поля во внешней среде. 

В силу линейной постановки задачи
\begin{equation}
\P\en=\P\o+\P_s,
\end{equation}
где $\P_s$ -- потенциал скоростей рассеянной звуковой волны.

Тогда из \eqref{eq_gel_for_enviroment} получаем уравнение для нахождения $\P_s:$
\begin{equation}\label{eq__gel_for_psi_s}
\Lap\P_s+k\en^2\P_s=0.
\end{equation}

Уравнения \eqref{eq_gel_for_polost} и \eqref{eq__gel_for_psi_s} запишем в цилиндрических системах координат $\r\c, \f\c, z\c$ и $\r\cyl, \f\cyl, z\cyl$ соответственно. 

Отраженная волна $\P_s$ должна удовлетворять условиям излучения на бесконечности, а звуковая волна в полости цилиндра $\P\c$ -- условию ограниченности.

Поэтому потенциалы $\P_s$ и $\P\c$ будем искать в виде
\todo{Беты одинаковые или разные (станет ясно в граничных условиях)}
\begin{align}
&\P_s(\r\cyl, \f\cyl, z\cyl)= \exp\{i\alpha z\cyl\}\sum_{n=-\infty}^{\infty}A_nH_n(\beta\cyl \r\cyl)\exp\{in\f\cyl\},\label{razloj_psi_s}\\
&\P\c(\r\c, \f\c, z\c)= \exp\{i\alpha z\c\}\sum_{n=-\infty}^{\infty}B_nH_n(\beta\c \r\c)\exp\{in\f\c\},\label{razloj_psi_o}
\end{align}
где $H_n(x)$ -- цилиндрическая функция Ханкеля первого рода порядка $n.$

Скорости частиц жидкости и акустические давления вне и внутри цилиндра определяются по следующим формулам соответственно:
\begin{align*}
&\bar{\nu}\c=\grad\P\c;\:\:\:\:\:P\c=i\r\c\omega\P\c,\\
&\bar{\nu}\en=\grad\P\en;\:\:\:\:\:P\en=i\r\en\:\omega\P\en.
\end{align*}

Распространение малых возмущений в упругом теле для установившегося режима движения частиц тела описывается скалярным и векторным уравнениями Гельмгольца:
\begin{align}
&\Lap\bar{\F}\cyl+k_{\popr}^2\bar{\F}\cyl=0,\\
&\Lap\P\cyl+k\prodol^2\P\cyl=0,\label{fi_gel}
\end{align}
где $k_{\popr}=\omega/c_{\popr}$ и $k\prodol=\omega/c\prodol$ -- волновые числа продольных и поперечных упругих волн соответственно; $\P\cyl$ и $\bar\F\cyl$ -- скалярный и векторный потенциалы смещения соответственно; $c\prodol=\sqrt{(\lambda\cyl+2\mu\cyl)/\r\cyl}$ и $c_{\popr}=\sqrt{\mu\cyl/\r\cyl}$ -- скорости продольных и поперечных волн соответственно.

При этом вектор смещения $\bar{u}\cyl$ представляется в виде:
\begin{equation}
\bar{u}\cyl=\grad\P\cyl+\rot\bar{\F}\cyl.
\end{equation}

Векторное уравнение \eqref{fi_gel} в цилиндрической системе координат в обшем случае не распадается на три независимых скалярных уравнения относительно проекций вектора $\bar{\F}\cyl,$ а представляет собой систему трех уравнений, решение которой сопряжено со значительными математическими трудностями.

Представим вектор $\bar\F\cyl$ в виде
$$\bar\F\cyl=\rot(L\bar{e}_{z\cyl})+\frac1{k_{\popr}}\rot\rot(M\bar{e}_{z\cyl})=\rot(L\bar{e}_{z\cyl})+k_{\popr} M\bar{e}_{z\cyl}+\frac1{k_{\popr}}\grad\biggl(\frac{\partial M}{\partial z\cyl}\biggr),$$
где $L$ и $M$ -- скалярные функции пространственных координат $\r\cyl, \varphi\cyl, z\cyl;$
\newline
$\bar{e}_{z\cyl}$ -- единичный вектор оси $z\cyl.$

Тогда векторное уравнение \eqref{fi_gel} заменится двумя скалярными уравнениями Гельмгольца относительно функций $L$ и $M$
\begin{align*}
&\Lap L+k_{\popr}^2L=0,\\
&\Lap M+k_{\popr}^2M=0.
\end{align*}

С учетом условия ограниченности функции $\P, L$ и $M$ будем искать в виде
\begin{align}
&\P\cyl(\r\cyl, \f\cyl, z\cyl)= \exp\{i\alpha z\cyl\}\sum_{n=-\infty}^{\infty}C_nH_n(\beta_{\prodol} \r\cyl)\exp\{in\f\cyl\},\label{pazloj_psi}\\
&L(\r\cyl, \f\cyl, z\cyl)= \exp\{i\alpha z\cyl\}\sum_{n=-\infty}^{\infty}D_nH_n(\beta_{\popr} \r\cyl)\exp\{in\f\cyl\},\label{pazloj_l}\\
&M(\r\cyl, \f\cyl, z\cyl)= \exp\{i\alpha z\cyl\}\sum_{n=-\infty}^{\infty}E_nH_n(\beta_{\popr} \r\cyl)\exp\{in\f\cyl\},\label{pazloj_m}
\end{align}
где $\beta_{\prodol}=\sqrt{k_{\prodol}^2-\alpha^2}, \beta_{\popr}=\sqrt{k_{\popr}^2-\alpha^2}.$

Компоненты вектора смещения $\bar{u}\cyl,$ записанные через функции $\P\cyl, L$ и $M$ в цилиндрической системе координат, имеют вид
\begin{align}\label{u_cherez_L_M}
&{u\cyl}_\r=\frac{\partial \P\cyl}{\partial\r\cyl}+\frac{\partial^2 L}{\partial\r\cyl\:\partial z\cyl}+\frac{k_{\popr}}{\r\cyl}\frac{\partial M}{\partial \f\cyl},\\
&{u\cyl}_\f=\frac{1}{\r\cyl}\frac{\partial \P\cyl}{\partial \f\cyl}+\frac{1}{\r\cyl}\frac{\partial^{2} L}{\partial\f\cyl\: \partial z\cyl}-k_{\popr}\frac{\partial M}{\partial \r\cyl},\\
&{u\cyl}_z=\frac{\partial\P\cyl}{\partial z\cyl}-\frac{\partial^{2} L}{\partial \r^2\cyl}-\frac{1}{\r\cyl}\frac{\partial L}{\partial \r\cyl}-\frac{1}{\r^2\cyl}\frac{\partial^2L}{\partial\f^2\cyl}.
\end{align}

Соотношения между компонентами тензора напряжений ${\sigma\cyl}_{ij}$ и вектора смещения $\bar{u}\cyl$ в однородном изотропном упругом цилиндре записываются следующим образом:
\todo{Здесь номер для системы уравнений находится внизу, а должен быть посередине системы}
\begin{equation}\label{tenzor_napr}
\begin{aligned}
&{\sigma\cyl}_{\r\r}=\lambda\cyl\Biggl(\frac{\partial {u\cyl}_{\r}}{\partial \r\cyl}+\frac{1}{\r\cyl}\biggl(\frac{\partial {u\cyl}_{\f}}{\partial\f\cyl}+{u\cyl}_{\r}\biggr)+\frac{\partial {u\cyl}_{z}}{\partial z\cyl}\Biggr)+2\:\mu\cyl\frac{\partial {u\cyl}_{\r}}{\partial \r\cyl},\\
&{\sigma\cyl}_{\f\f}=\lambda\cyl\Biggl(\frac{\partial {u\cyl}_{\r}}{\partial \r\cyl}+\frac{1}{\r\cyl}\biggl(\frac{\partial {u\cyl}_{\f}}{\partial\f\cyl}+{u\cyl}_{\r}\biggr)+\frac{\partial {u\cyl}_{z}}{\partial z\cyl}\Biggr)+2\:\mu\cyl\biggl(\frac{1}{\r\cyl}\frac{\partial {u\cyl}_{\f}}{\partial\f\cyl}+\frac{{u\cyl}_{\r}}{\r\cyl}\biggr),\\
&{\sigma\cyl}_{zz}=\lambda\cyl\Biggl(\frac{\partial {u\cyl}_{\r}}{\partial \r\cyl}+\frac{1}{\r\cyl}\biggl(\frac{\partial {u\cyl}_{\f}}{\partial\f\cyl}+{u\cyl}_{\r}\biggr)+\frac{\partial {u\cyl}_{z}}{\partial z\cyl}\Biggr)+2\:\mu\cyl\frac{\partial {u\cyl}_{z}}{\partial z\cyl},\\
&{\sigma\cyl}_{\r\f}=\mu\cyl\Biggl(\frac{1}{\r\cyl}\frac{\partial {u\cyl}_{\r}}{\partial\f\cyl}+\frac{\partial {u\cyl}_{\f}}{\partial \r\cyl}-\frac{{u\cyl}_{\f}}{\r\cyl}\Biggr),\\
&{\sigma\cyl}_{\r z}=\mu\cyl\Biggl(\frac{\partial {u\cyl}_{z}}{\partial \r\cyl}+\frac{\partial {u\cyl}_{\r}}{\partial z\cyl}\Biggr),\\
&{\sigma\cyl}_{\f z}=\mu\cyl\Biggl(\frac{\partial {u\cyl}_{\f}}{\partial z\cyl}+\frac{1}{\r\cyl}\frac{\partial {u\cyl}_{z}}{\partial \f\cyl}\Biggr).
\end{aligned}
\end{equation}

Уравнения движения неоднородного изотропного упругого цилиндрического слоя в случае установившихся колебаний в цилиндрической системе координат имеют вид:
\begin{equation}\label{equation_dvij_splosh_sred}
\begin{aligned}
\frac{\partial\:{\sigma\l}_{\r\r}}{
\partial \r\cyl}+\frac{1}{\r\cyl}\:\frac{\partial\:{\sigma\l}_{\r\f}}{\partial\f\cyl}+\frac{\partial\:{\sigma\l}_{\r z}}{\partial z\cyl}+\frac{{\sigma\l}_{\r\r}-{\sigma\l}_{\f\f}}{\r\cyl}&=-\omega^2 p\l{u\l}_{\r},\\
\frac{\partial\:{\sigma\l}_{\r\f}}{
\partial \r\cyl}+\frac{1}{\r\cyl}\:\frac{\partial\:{\sigma\l}_{\f\f}}{\partial\f\cyl}+\frac{\partial\:{\sigma\l}_{\f z}}{\partial z\cyl}+\frac{2}{\r\cyl}\:{\sigma\l}_{\r\f}&=-\omega^2 p\l{u\l}_{\f},\\
\frac{\partial\:{\sigma\l}_{\r z}}{
\partial \r\cyl}+\frac{1}{\r\cyl}\:\frac{\partial\:{\sigma\l}_{\f z}}{\partial\f\cyl}+\frac{\partial\:{\sigma\l}_{zz}}{\partial z\cyl}+\frac{1}{\r\cyl}\:{\sigma\l}_{\r z}&=-\omega^2 p\l{u\l}_{z},
\end{aligned}
\end{equation}
где ${u\l}_{\r}, {u\l}_{\f}, {u\l}_{z}$ -- компоненты вектора смещения $\bar{u}\l$ частиц неоднородного слоя; 
\newline
${\sigma\l}_{ij}$ -- компоненты тензора напряжений в неоднородном слое.

Соотношения между компонентами тензора напряжений ${\sigma\l}_{ij}$ и вектора смещения $\bar{u}\l$ в неоднородном упругом цилиндрическом слое аналогичны соотношениям \eqref{tenzor_napr} для однородного упругого цилиндра, при этом упругие постоянные $\lambda\cyl$ и $\mu\cyl$ следует заменить на функции $\lambda\l=\lambda\l(\r\cyl)$ и $\mu\l=\mu\l(\r\cyl).$

Используя эти соотношения, запишем уравнения \eqref{equation_dvij_splosh_sred} через компоненты вектора смещения $\bar{u}\l.$ Получим
\begin{equation}\label{equation_otn_u}
\begin{aligned}
&\biggl(\lambda\l+2\mu\l\biggr)\frac{\partial^2{u\l}_{\r}}{\partial \r^2\cyl}+\biggl(\frac{\partial\lambda\l}{\partial \r\cyl}+2\:\frac{\partial\mu\l}{\partial \r\cyl}+\frac{\lambda\l+2\mu\l}{\r\cyl}\biggr)\frac{\partial {u\l}_{\r}}{\partial \r\cyl}+\frac{\mu\l}{\r^2\cyl}\frac{\partial^2 {u\l}_{\r}}{\partial \f^2\cyl}+\\
&+\mu\l\:\frac{\partial^2 {u\l}_{\r}}{\partial z^2\cyl}+\frac{\lambda\l+\mu\l}{\r\cyl}\frac{\partial^2 {u\l}_{\f}}{\partial \r\cyl\partial\f\cyl}+\frac{1}{\r\cyl}\biggl(\frac{\partial\lambda\l}{\partial \r\cyl}-\frac{\lambda\l+3\mu\l}{\r\cyl}\biggr)\frac{\partial {u\l}_{\f}}{\partial\f\cyl}+\\
&+\biggl(\lambda\l+\mu\l\biggr)\frac{\partial^2 {u\l}_{z}}{\partial \r\cyl\partial z\cyl}+\frac{\partial\lambda\l}{\partial \r\cyl}\:\frac{\partial {u\l}_{z}}{\partial z\cyl}+\biggl(\frac{1}{\r\cyl}\:\frac{\partial\lambda\l}{\partial \r\cyl}-\frac{\lambda\l+2\mu\l}{\r^2\cyl}+\om^2 p\l\biggr){u\l}_{\r}=0,\\
&\frac{\lambda\l+\mu\l}{\r\cyl}\frac{\partial^2{u\l}_{\r}}{\partial \r\cyl\partial\f\cyl}+\frac{1}{\r\cyl}\biggl(\frac{\partial\mu\l}{\partial \r\cyl}+\frac{\lambda\l+3\:\mu\l}{\r\cyl}\biggr)\frac{\partial {u\l}_{\r}}{\partial\f\cyl}+\mu\l\:\frac{\partial^2 {u\l}_{\f}}{\partial \r^2\cyl}+\\
&+\frac{\lambda\l+2\:\mu\l}{\r^2\cyl}\:\frac{\partial^2 {u\l}_{\f}}{\partial \f^2\cyl}+\biggl(\frac{\partial\mu\l}{\partial \r\cyl}+\frac{\mu\l}{\r\cyl}\biggr)\frac{\partial {u\l}_{\f}}{\partial \r\cyl}+\mu\l\:\frac{\partial^2 {u\l}_{\f}}{\partial z^2\cyl}+\\
&+\frac{\lambda\l+\mu\l}{\r\cyl}\frac{\partial^2 {u\l}_{z}}{\partial \f\cyl\partial z\cyl}+\biggl(-\frac{1}{\r\cyl}\:\frac{\partial\mu\l}{\partial \r\cyl}-\frac{\mu\l}{\r^2\cyl}+\om^2 p\l\biggr){u\l}_{\f}=0,\\
&\biggl(\lambda\l+\mu\l\biggr)\frac{\partial^2{u\l}_{\r}}{\partial \r\cyl\partial z\cyl}+\biggl(\frac{\partial\mu\l}{\partial \r\cyl}+\frac{\lambda\l+\mu\l}{\r\cyl}\biggr)\frac{\partial {u\l}_{\r}}{\partial z\cyl}+\frac{\lambda\l+\mu\l}{\r\cyl}\:\frac{\partial^2 {u\l}_{\f}}{\partial \f\cyl\partial z\cyl}+\\
&+\mu\l\:\frac{\partial^2 {u\l}_{z}}{\partial \r^2\cyl}+\frac{\mu\l}{\r^2\cyl}\:\frac{\partial^2 {u\l}_{z}}{\partial {\f\cyl}^2}+\biggl(\lambda\l+2\:\mu\l\biggr)\frac{\partial^2 {u\l}_{z}}{\partial z^2\cyl}+\\
&+\biggl(\frac{\partial\mu\l}{\partial \r\cyl}+\frac{\mu\l}{\r\cyl}\biggr)\frac{\partial {u\l}_{z}}{\partial \r\cyl}+\om^2 p\l\:{u\l}_{z}=0.
\end{aligned}
\end{equation}

\newpage
Функции ${u\l}_{\r}(\r\cyl, \f\cyl, z\cyl),$ ${u\l}_{\f}(\r\cyl, \f\cyl, z\cyl)$ и ${u\l}_{z}(\r\cyl, \f\cyl, z\cyl)$ будем искать в виде разложений
\begin{equation}\label{razloj_u}
\begin{aligned}
{u\l}_{\r}(\r\cyl, \f\cyl, z\cyl)&=\exp\{i\alpha z\cyl\}\sum_{n=-\infty}^{\infty}{u\l}_{\r\:n}(\r\cyl)\exp\{in\f\cyl\},\\
{u\l}_{\f}(\r\cyl, \f\cyl, z\cyl)&=\exp\{i\alpha z\cyl\}\sum_{n=-\infty}^{\infty}{u\l}_{\f\:n}(\r\cyl)\exp\{in\f\cyl\},\\
{u\l}_{z}(\r\cyl, \f\cyl, z\cyl)&=\exp\{i\alpha z\cyl\}\sum_{n=-\infty}^{\infty}{u\l}_{z\:n}(\r\cyl)\exp\{in\f\cyl\}.
\end{aligned}
\end{equation}

Коэффициенты $A_n,$ $B_n,$ $C_n,$ $D_n,$ $E_n$ разложений \eqref{razloj_psi_s}, \eqref{razloj_psi_o}, \eqref{razloj_psi_s}, \eqref{pazloj_psi}, \eqref{pazloj_l}, \eqref{pazloj_m} и функции ${u\l}_{\r\:n}(\r\cyl),$ ${u\l}_{\f\:n}(\r\cyl)$ и ${u\l}_{z\:n}(\r\cyl)$ из разложений \eqref{razloj_u} подлежат определению из граничных условий.

Граничные условия на внешней поверхности слоя заключаются в равенстве нормальных скоростей частиц упругой неоднородной среды и жидкости, равенстве на ней нормального напряжения и акустического давления, отсутствии касательных напряжений:
\begin{equation}\label{vneshnaja_pover_sloja}
\begin{aligned}
\r\cyl &=R\l: &-iw{u\l}_{\r}&={v\en}_{\r}, &{\sigma\l}_{\r\r}&=-P\en, &{\sigma\l}_{\r\f}&=0, &{\sigma\l}_{\r z}&=0.
\end{aligned}
\end{equation}

На внутренней поверхности слоя при переходе через границу раздела упругих сред должны быть непрерывны составляющие вектора смещения частиц, а также нормальные и тангенциальные напряжения:
\begin{equation}\label{vnutr_pover_sloja}
\begin{aligned}
\r\cyl&=R\cyl: &{u\l}_{\r}&={u\cyl}_{\r}, &{u\l}_{\f}&={u\cyl}_{\f}, &{u\l}_{z}&={u\cyl}_{z},\\
& &{\sigma\l}_{\r\r}&={\sigma\cyl}_{\r\r}, &{\sigma\l}_{\r\f}&={\sigma\cyl}_{\r\f},  &{\sigma\l}_{\r z}&={\sigma\cyl}_{\r z}.
\end{aligned}
\end{equation}

На границе полости $\r\c=R\c$ должны выполняться граничные условия, заключающиеся в отсутствии нормальных и тангециальных составляющих тензора напряжений
\begin{equation}\label{gran_pol}
\begin{aligned}
\r\c &=R\c: &-iw{u\cyl}_{\r}&={v\c}_{\r}, &{\sigma\cyl}_{\r\r}&=-P\c, &{\sigma\cyl}_{\r\f}&=0, &{\sigma\cyl}_{\r z}&=0.
\end{aligned}
\end{equation}

Используя формулы
$${v\en}_{\r}=\frac{\de(\Psi_{o}+\Psi_s)}{\de\r\cyl},\:\:\: P\en=iwp\en(\Psi_{o}+\Psi_s)$$
и выражение \eqref{tenzor_napr} запишем граничные условия \eqref{vneshnaja_pover_sloja} через функции $\Psi_{o},$ $\Psi_s,$ ${u\l}_\r,$ ${u\l}_\f,$ ${u\l}_z$. Получим при $\r\cyl=R\l$
\begin{align}
&\frac{\de(\Psi_{o}+\Psi_s)}{\de\r\cyl}=-iw{u\l}_{\r},\\
&\lambda\l\Biggl(\frac{\partial {u\l}_{\r}}{\partial \r\l}+\frac{1}{\r\l}\biggl(\frac{\partial {u\l}_\f}{\partial\f\l}+{u\l}_{\r}\biggr)+\frac{\partial {u\l}_z}{\partial z\l}\Biggr)+2\:\mu\l\:\frac{\partial {u\l}_{\r}}{\partial \r\l}=-iwp\en(\Psi_{o}+\Psi_s),\\
&\mu\l\Biggl(\frac{1}{\r\l}\frac{\partial {u\l}_{\r}}{\partial\f\l}+\frac{\partial {u\l}_{\f}}{\partial \r\l}-\frac{{u\l}_{\f}}{\r\l}\Biggr)=0,\\
&\mu\l\Biggl(\frac{\partial {u\l}_z}{\partial \r\l}+\frac{\partial {u\l}_{\r}}{\partial z\l}\Biggr)=0.\\
\end{align}

\todo{Исправить фразу}
Аналогично, используя выражения \eqref{u_cherez_L_M}, \eqref{tenzor_napr} запишем граничные условия \eqref{vnutr_pover_sloja} через функции $u_\r,$ $u_\f,$ $u_z,$ $ \P\cyl,$ $L$ и $M ,$ а граничные условия \eqref{vnutr_pover_sloja} через $\P\cyl,$ $L$ и $M.$

\begin{align}
&{u\l}_\r=\frac{\partial \P\cyl}{\partial\r\cyl}+\frac{\partial^2 L}{\partial\r\cyl\:\partial z\cyl}+\frac{k_{\popr}}{\r\cyl}\frac{\partial M}{\partial \f\cyl},\\
&{u\l}_\f=\frac{1}{\r\cyl}\frac{\partial \P\cyl}{\partial \f\cyl}+\frac{1}{\r\cyl}\frac{\partial^{2} L}{\partial\f\cyl\: \partial z\cyl}-k_{\popr}\frac{\partial M}{\partial \r\cyl},\\
&{u\l}_z=\frac{\partial\P\cyl}{\partial z\cyl}-\frac{\partial^{2} L}{\partial \r^2\cyl}-\frac{1}{\r\cyl}\frac{\partial L}{\partial \r\cyl}-\frac{1}{\r^2\cyl}\frac{\partial^2L}{\partial\f^2\cyl}.\\
&\lambda\l\Biggl(\frac{\partial {u\l}_{\r}}{\partial \r\l}+\frac{1}{\r\l}\biggl(\frac{\partial {u\l}_\f}{\partial\f\l}+{u\l}_{\r}\biggr)+\frac{\partial {u\l}_z}{\partial z\l}\Biggr)+2\:\mu\l\:\frac{\partial {u\l}_{\r}}{\partial \r\l}=\\
&\lambda\cyl\Biggl(\frac{\partial {u\cyl}_{\r}}{\partial \r\cyl}+\frac{1}{\r\cyl}\biggl(\frac{\partial {u\cyl}_{\f}}{\partial\f\cyl}+{u\cyl}_{\r}\biggr)+\frac{\partial {u\cyl}_{z}}{\partial z\cyl}\Biggr)+2\:\mu\cyl\frac{\partial {u\cyl}_{\r}}{\partial \r\cyl},\\
&\mu\l\Biggl(\frac{1}{\r\l}\frac{\partial {u\l}_{\r}}{\partial\f\l}+\frac{\partial {u\l}_{\f}}{\partial \r\l}-\frac{{u\l}_{\f}}{\r\l}\Biggr)=\\
&=\mu\cyl\Biggl(\frac{1}{\r\cyl}\frac{\partial {u\cyl}_{\r}}{\partial\f\cyl}+\frac{\partial {u\cyl}_{\f}}{\partial \r\cyl}-\frac{{u\cyl}_{\f}}{\r\cyl}\Biggr),\\
&\mu\l\Biggl(\frac{\partial {u\l}_z}{\partial \r\l}+\frac{\partial {u\l}_{\r}}{\partial z\l}\Biggr)=\\
&=\mu\cyl\Biggl(\frac{\partial {u\cyl}_{z}}{\partial \r\cyl}+\frac{\partial {u\cyl}_{\r}}{\partial z\cyl}\Biggr).
\end{align}

Подставив разложения \eqref{razloj_u} в уравнения \eqref{equation_otn_u}, воспользовавшись уравнением для присоединенных многочленов Лежандра и свойством ортогональности этих многочленов, получим для каждого индекса $n \q(n = 0,1,\ldots;$ систему линейных однородных обыкновенных дифференциальных уравнений второго порядка относительно неизвестных функций $u_{\r\:n}(\r\cyl),$ $u_{\f\:n}(\r\cyl)$ и $u_{z\:n}(\r\cyl).$ А подставив разложения \eqref{razloj_u} в полученные граничные условия, найдем краевые условия и сможем решить  систему дифференциальных уравнений. После ее решения определим коэффициенты $A_n,$ $B_n,$ $C_n,$ $D_n,$ $E_n$ разложений  \eqref{razloj_psi_s}, \eqref{razloj_psi_o}, \eqref{razloj_psi_s}, \eqref{pazloj_psi}, \eqref{pazloj_l}, \eqref{pazloj_m} для каждой индекса $n,$ а зная коэффициенты ${A}_{n}$, по формуле \eqref{razloj_psi_s} найдем акустическое поле, рассеяное упругим цилиндром, имеющим произвольно расположенную полость и неоднородное покрытие.
\newpage
\subsection{Решение краевой задачи для системы обыкновенных дифференциальных уравнений}

\newpage
\section{ЧИСЛЕННЫЕ ИССЛЕДОВАНИЯ}

\newpage
\subsection{Диаграмма направленности}

\newpage
\subsection{Частотные характеристики}


\newpage
\section*{ЗАКЛЮЧЕНИЕ}
\addcontentsline{toc}{section}{ЗАКЛЮЧЕНИЕ}
Была поставлена задача о~дифракции плоских звуковых волн на~упругой сфере, имеющей произвольно расположенную полость и неоднородное покрытие. В~данной работе приведены основные уравнения колебаний, а также разложения в~ряд искомых функций для~внешней среды сферы, а также полости тела.

\newpage
\section*{ЛИТЕРАТУРА}
\addcontentsline{toc}{section}{ЛИТЕРАТУРА}

\newpage
\section*{ПРИЛОЖЕНИЕ}
\addcontentsline{toc}{section}{ПРИЛОЖЕНИЕ}
\end{document}